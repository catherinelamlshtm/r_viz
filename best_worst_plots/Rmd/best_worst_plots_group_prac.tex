% Options for packages loaded elsewhere
\PassOptionsToPackage{unicode}{hyperref}
\PassOptionsToPackage{hyphens}{url}
%
\documentclass[
]{article}
\usepackage{amsmath,amssymb}
\usepackage{lmodern}
\usepackage{iftex}
\ifPDFTeX
  \usepackage[T1]{fontenc}
  \usepackage[utf8]{inputenc}
  \usepackage{textcomp} % provide euro and other symbols
\else % if luatex or xetex
  \usepackage{unicode-math}
  \defaultfontfeatures{Scale=MatchLowercase}
  \defaultfontfeatures[\rmfamily]{Ligatures=TeX,Scale=1}
\fi
% Use upquote if available, for straight quotes in verbatim environments
\IfFileExists{upquote.sty}{\usepackage{upquote}}{}
\IfFileExists{microtype.sty}{% use microtype if available
  \usepackage[]{microtype}
  \UseMicrotypeSet[protrusion]{basicmath} % disable protrusion for tt fonts
}{}
\makeatletter
\@ifundefined{KOMAClassName}{% if non-KOMA class
  \IfFileExists{parskip.sty}{%
    \usepackage{parskip}
  }{% else
    \setlength{\parindent}{0pt}
    \setlength{\parskip}{6pt plus 2pt minus 1pt}}
}{% if KOMA class
  \KOMAoptions{parskip=half}}
\makeatother
\usepackage{xcolor}
\usepackage[margin=1in]{geometry}
\usepackage{color}
\usepackage{fancyvrb}
\newcommand{\VerbBar}{|}
\newcommand{\VERB}{\Verb[commandchars=\\\{\}]}
\DefineVerbatimEnvironment{Highlighting}{Verbatim}{commandchars=\\\{\}}
% Add ',fontsize=\small' for more characters per line
\usepackage{framed}
\definecolor{shadecolor}{RGB}{248,248,248}
\newenvironment{Shaded}{\begin{snugshade}}{\end{snugshade}}
\newcommand{\AlertTok}[1]{\textcolor[rgb]{0.94,0.16,0.16}{#1}}
\newcommand{\AnnotationTok}[1]{\textcolor[rgb]{0.56,0.35,0.01}{\textbf{\textit{#1}}}}
\newcommand{\AttributeTok}[1]{\textcolor[rgb]{0.77,0.63,0.00}{#1}}
\newcommand{\BaseNTok}[1]{\textcolor[rgb]{0.00,0.00,0.81}{#1}}
\newcommand{\BuiltInTok}[1]{#1}
\newcommand{\CharTok}[1]{\textcolor[rgb]{0.31,0.60,0.02}{#1}}
\newcommand{\CommentTok}[1]{\textcolor[rgb]{0.56,0.35,0.01}{\textit{#1}}}
\newcommand{\CommentVarTok}[1]{\textcolor[rgb]{0.56,0.35,0.01}{\textbf{\textit{#1}}}}
\newcommand{\ConstantTok}[1]{\textcolor[rgb]{0.00,0.00,0.00}{#1}}
\newcommand{\ControlFlowTok}[1]{\textcolor[rgb]{0.13,0.29,0.53}{\textbf{#1}}}
\newcommand{\DataTypeTok}[1]{\textcolor[rgb]{0.13,0.29,0.53}{#1}}
\newcommand{\DecValTok}[1]{\textcolor[rgb]{0.00,0.00,0.81}{#1}}
\newcommand{\DocumentationTok}[1]{\textcolor[rgb]{0.56,0.35,0.01}{\textbf{\textit{#1}}}}
\newcommand{\ErrorTok}[1]{\textcolor[rgb]{0.64,0.00,0.00}{\textbf{#1}}}
\newcommand{\ExtensionTok}[1]{#1}
\newcommand{\FloatTok}[1]{\textcolor[rgb]{0.00,0.00,0.81}{#1}}
\newcommand{\FunctionTok}[1]{\textcolor[rgb]{0.00,0.00,0.00}{#1}}
\newcommand{\ImportTok}[1]{#1}
\newcommand{\InformationTok}[1]{\textcolor[rgb]{0.56,0.35,0.01}{\textbf{\textit{#1}}}}
\newcommand{\KeywordTok}[1]{\textcolor[rgb]{0.13,0.29,0.53}{\textbf{#1}}}
\newcommand{\NormalTok}[1]{#1}
\newcommand{\OperatorTok}[1]{\textcolor[rgb]{0.81,0.36,0.00}{\textbf{#1}}}
\newcommand{\OtherTok}[1]{\textcolor[rgb]{0.56,0.35,0.01}{#1}}
\newcommand{\PreprocessorTok}[1]{\textcolor[rgb]{0.56,0.35,0.01}{\textit{#1}}}
\newcommand{\RegionMarkerTok}[1]{#1}
\newcommand{\SpecialCharTok}[1]{\textcolor[rgb]{0.00,0.00,0.00}{#1}}
\newcommand{\SpecialStringTok}[1]{\textcolor[rgb]{0.31,0.60,0.02}{#1}}
\newcommand{\StringTok}[1]{\textcolor[rgb]{0.31,0.60,0.02}{#1}}
\newcommand{\VariableTok}[1]{\textcolor[rgb]{0.00,0.00,0.00}{#1}}
\newcommand{\VerbatimStringTok}[1]{\textcolor[rgb]{0.31,0.60,0.02}{#1}}
\newcommand{\WarningTok}[1]{\textcolor[rgb]{0.56,0.35,0.01}{\textbf{\textit{#1}}}}
\usepackage{graphicx}
\makeatletter
\def\maxwidth{\ifdim\Gin@nat@width>\linewidth\linewidth\else\Gin@nat@width\fi}
\def\maxheight{\ifdim\Gin@nat@height>\textheight\textheight\else\Gin@nat@height\fi}
\makeatother
% Scale images if necessary, so that they will not overflow the page
% margins by default, and it is still possible to overwrite the defaults
% using explicit options in \includegraphics[width, height, ...]{}
\setkeys{Gin}{width=\maxwidth,height=\maxheight,keepaspectratio}
% Set default figure placement to htbp
\makeatletter
\def\fps@figure{htbp}
\makeatother
\setlength{\emergencystretch}{3em} % prevent overfull lines
\providecommand{\tightlist}{%
  \setlength{\itemsep}{0pt}\setlength{\parskip}{0pt}}
\setcounter{secnumdepth}{-\maxdimen} % remove section numbering
\newlength{\cslhangindent}
\setlength{\cslhangindent}{1.5em}
\newlength{\csllabelwidth}
\setlength{\csllabelwidth}{3em}
\newlength{\cslentryspacingunit} % times entry-spacing
\setlength{\cslentryspacingunit}{\parskip}
\newenvironment{CSLReferences}[2] % #1 hanging-ident, #2 entry spacing
 {% don't indent paragraphs
  \setlength{\parindent}{0pt}
  % turn on hanging indent if param 1 is 1
  \ifodd #1
  \let\oldpar\par
  \def\par{\hangindent=\cslhangindent\oldpar}
  \fi
  % set entry spacing
  \setlength{\parskip}{#2\cslentryspacingunit}
 }%
 {}
\usepackage{calc}
\newcommand{\CSLBlock}[1]{#1\hfill\break}
\newcommand{\CSLLeftMargin}[1]{\parbox[t]{\csllabelwidth}{#1}}
\newcommand{\CSLRightInline}[1]{\parbox[t]{\linewidth - \csllabelwidth}{#1}\break}
\newcommand{\CSLIndent}[1]{\hspace{\cslhangindent}#1}
  
\usepackage{amsmath}
\usepackage{siunitx}
\renewcommand\vec{\boldsymbol}
\def\begincols{\begin{columns}}
\def\begincol{\begin{column}}
\def\endcol{\end{column}}
\def\endcols{\end{columns}}
\makeatletter
\renewcommand*\env@matrix[1][*\c@MaxMatrixCols c]{%
  \hskip -\arraycolsep
  \let\@ifnextchar\new@ifnextchar
  \array{#1}}
\makeatother
\DeclareSIUnit\horsepower{hp}
\usepackage{xstring}
\usepackage{float}
\renewcommand{\vec}[1]{\mathbf{#1}}
\usepackage{booktabs}
\usepackage[version=4]{mhchem}
\newcommand{\derivn}[3]{\frac{\mathrm{d}^{#3}#1}{\mathrm{d}#2 ^{#3}}}
\newcommand{\deriv}[2]{\frac{\mathrm{d}#1}{\mathrm{d}#2}}
\newcommand{\dx}[1]{\,\mathrm{d}#1}
\usepackage{sourcecodepro}
%\usepackage{sourcesanspro}
%\usepackage{sourceserifpro}
\usepackage[sfdefault, condensed]{roboto}
\usepackage{xcolor}
\hypersetup{colorlinks=true, allcolors=links}
\definecolor{links}{HTML}{7E6DE0}


\usepackage{longtable}
\ifLuaTeX
  \usepackage{selnolig}  % disable illegal ligatures
\fi
\IfFileExists{bookmark.sty}{\usepackage{bookmark}}{\usepackage{hyperref}}
\IfFileExists{xurl.sty}{\usepackage{xurl}}{} % add URL line breaks if available
\urlstyle{same} % disable monospaced font for URLs
\hypersetup{
  pdftitle={Visualisation with ggplot2},
  pdfauthor={Sam Clifford},
  hidelinks,
  pdfcreator={LaTeX via pandoc}}

\title{Visualisation with ggplot2}
\usepackage{etoolbox}
\makeatletter
\providecommand{\subtitle}[1]{% add subtitle to \maketitle
  \apptocmd{\@title}{\par {\large #1 \par}}{}{}
}
\makeatother
\subtitle{2031 - Introduction to Statistical Computing}
\author{Sam Clifford}
\date{2020-11-26}

\begin{document}
\maketitle

\hypertarget{introduction}{%
\section{Introduction}\label{introduction}}

\hypertarget{about-this-practical-session}{%
\subsection{About this practical
session}\label{about-this-practical-session}}

In the lecture session we introduced visualisation with the histogram,
\(x\)-\(y\) plots and other scatter plot techniques, and touched on
Tufte's principles of graphical excellence.

This prac will investigate the visual display of data and what makes a
good and a bad graph. You will first critique a plot produced through
the provided code. Then, you may choose which order to do activities 2a
and 2b, which are to produce a better and a worse version of the given
plot. You should finish at least one of them during the time given.

\begin{itemize}
\tightlist
\item
  Assumed skills

  \begin{itemize}
  \tightlist
  \item
    Writing R code into a script file
  \item
    Identifying things that are visually pleasing
  \end{itemize}
\item
  Learning objectives

  \begin{itemize}
  \tightlist
  \item
    Identifying things that are informative
  \item
    Being able to critique a graph
  \item
    Understanding why and how data is encoded and decoded visually
  \item
    Understanding the subjectivity of what is aesthetically pleasing
  \end{itemize}
\item
  Professional skills

  \begin{itemize}
  \tightlist
  \item
    Creating high quality graphics
  \end{itemize}
\end{itemize}

\hypertarget{group-formation}{%
\subsection{Group formation}\label{group-formation}}

You will be allocated to groups of 3 in Zoom breakout rooms.

A reminder of expectations in the prac:

\begin{itemize}
\tightlist
\item
  Keep a record of the work being completed with a well-commented R
  script
\item
  Allow everyone a chance to participate in the learning activities,
  keeping disruption of other students to a minimum while still allowing
  for fruitful discussion
\item
  All opinions are valued provided they do not harm others
\item
  Everyone is expected to do the work, learning seldom occurs solely by
  watching someone else do work
\end{itemize}

\hypertarget{activity-1---building-an-attempt-at-a-plot}{%
\section{Activity 1 - Building an attempt at a
plot}\label{activity-1---building-an-attempt-at-a-plot}}

We will be looking at the gapminder data set as found in the gapminder
package (Bryan 2017). This data has been collected from countries around
the world and contains data on life expectancy, population and GDP per
capita for 142 countries from 1952 to 2007.

\textbf{Exercise:} Copy and paste the code below to produce a plot
showing how the relationship between GDP, life expectancy and population
vary over time and continent. If you can't install the gapminder
package, you can download the data from Moodle and load it with
\texttt{read\_csv()} from the readr package (loaded when tidyverse is
loaded).

\small

\begin{Shaded}
\begin{Highlighting}[]
\FunctionTok{library}\NormalTok{(gapminder)}
\FunctionTok{library}\NormalTok{(tidyverse)}
\FunctionTok{data}\NormalTok{(gapminder)}

\FunctionTok{ggplot}\NormalTok{(}\AttributeTok{data =}\NormalTok{ gapminder,}
       \FunctionTok{aes}\NormalTok{(}\AttributeTok{x =}\NormalTok{ gdpPercap, }\AttributeTok{y =}\NormalTok{ lifeExp)) }\SpecialCharTok{+}
  \FunctionTok{geom\_path}\NormalTok{(}\FunctionTok{aes}\NormalTok{(}\AttributeTok{group =}\NormalTok{ country, }\AttributeTok{color =}\NormalTok{ continent)) }\SpecialCharTok{+}
  \FunctionTok{geom\_point}\NormalTok{(}\FunctionTok{aes}\NormalTok{(}\AttributeTok{color =}\NormalTok{ continent, }\AttributeTok{size =}\NormalTok{ pop)) }\SpecialCharTok{+}
  \FunctionTok{scale\_color\_brewer}\NormalTok{(}\AttributeTok{palette =} \StringTok{\textquotesingle{}Set2\textquotesingle{}}\NormalTok{) }\SpecialCharTok{+} \FunctionTok{scale\_x\_log10}\NormalTok{() }\SpecialCharTok{+}
  \FunctionTok{annotation\_logticks}\NormalTok{(}\AttributeTok{sides =} \StringTok{\textquotesingle{}b\textquotesingle{}}\NormalTok{) }\SpecialCharTok{+}
  \FunctionTok{facet\_wrap}\NormalTok{( }\SpecialCharTok{\textasciitilde{}}\NormalTok{ continent, }\AttributeTok{scales =} \StringTok{\textquotesingle{}free\textquotesingle{}}\NormalTok{, }\AttributeTok{nrow =} \DecValTok{1}\NormalTok{) }\SpecialCharTok{+}
  \FunctionTok{scale\_size\_area}\NormalTok{() }\SpecialCharTok{+}
  \FunctionTok{theme\_dark}\NormalTok{() }\SpecialCharTok{+}
  \FunctionTok{theme}\NormalTok{(}\AttributeTok{legend.position =} \StringTok{\textquotesingle{}bottom\textquotesingle{}}\NormalTok{,}
        \AttributeTok{legend.box =} \StringTok{\textquotesingle{}vertical\textquotesingle{}}\NormalTok{)}
\end{Highlighting}
\end{Shaded}

\begin{center}\includegraphics{best_worst_plots_group_prac_files/figure-latex/unnamed-chunk-2-1} \end{center}

\normalsize

\normalsize

\textbf{Exercise:} Discuss, within your group, what you think is good
and bad about this plot. Does it conform to Tufte's principles of
graphical excellence? Is it easy to interpret? Does it show the
relationship we are interested in? List \emph{three} important
improvements that are needed for this graph to be useful. This should
take no longer than five minutes.

\textbf{Exercise:} As a group, discuss what you think each line of code
in the above block does. You may wish to answer as comments in your code
(everything after a \texttt{\#} is a comment) or in a separate document.

\hypertarget{activity-2a-making-a-better-graph}{%
\section{Activity 2a -- Making a better
graph}\label{activity-2a-making-a-better-graph}}

Based on the ideas discussed, build a graph which your group believes
better shows the relationship between life expectancy and GDP. Think
first about what story you want your plot to tell; are you interested in
trends over space and/or time? Are you interested in a particular
continent or even just one country?

You may choose to either modify the code given above or create your own
graph from scratch. Make sure your code is written in your script file
with appropriate comments.

Some things you may wish to consider:

\begin{itemize}
\tightlist
\item
  fixing up the axis labels
\item
  a relevant title
\item
  a different theme
\item
  different plotting geometries
\item
  different aesthetic options for colour, shape, etc.
\end{itemize}

You may wish to sketch the graph by hand before attempting to write the
R code to generate it. This will help you and your group come to an
agreement about the plot you want to make and will help the tutors
understand what you're aiming for when you ask them for help.

If you get stuck, look at the
\href{https://ggplot2.tidyverse.org/reference/}{ggplot2} documentation
or ask a tutor.

\textbf{Exercise:} Make a plot, save it to your computer and write
comments in your code or standalone document that outline what the
changes you made were and why.

\hypertarget{activity-2b-making-a-worse-graph}{%
\section{Activity 2b -- Making a worse
graph}\label{activity-2b-making-a-worse-graph}}

Make a new graph as in the previous activity but make it as bad as
possible while still attempting to honestly show the information
(i.e.~don't add things to the plot which can't be derived from the
variables in the plot). Your plot should be an honest attempt to show
the data poorly, rather than a deliberately unreadable mess. Think of
something you'd expect to see in a newspaper staffed with
well-intentioned but unskilled staff.

Consider the principles of graphical excellence and how can we go
against them to make a truly terrible plot. Think about what was bad
about the plot provided earlier. Consider abusing the ability to map
graphical options (e.g.~color, fill, line type, point size) to our
variables of interest.

\textbf{Exercise:} Make a plot, save it to your computer and write
comments in your code or standalone document that outline what the
changes you made were and why.

\hypertarget{activity-3-group-discussion}{%
\section{Activity 3 -- Group
discussion}\label{activity-3-group-discussion}}

Have participants present their best and/or worst graph from the last
activities. What did they identify as good and bad and how has each
group attempted to present the relationship?

\hypertarget{tidy-up}{%
\section{Tidy up}\label{tidy-up}}

Make sure you save your R script, and anything else you have produced
and ensure everyone in your group has a copy. Email your worst graph to
\href{mailto:sam.clifford@lshtm.ac.uk}{Dr Sam Clifford}.

\hypertarget{further-reading}{%
\section{Further reading}\label{further-reading}}

A lot of the key ideas in data visualisation arose with Tufte (1983),
and are summarised by Pantoliano (2012). Some of the history of data
visualisation is summarised well by Friendly (2005) and Friendly (2006).
\href{https://www.edwardtufte.com/tufte/}{Tufte's website} is well worth
exploring, particularly the discussion on how the visual presentation of
information could have helped avert the \emph{Challenger} disaster
(Tufte 1997). For some more guidance on using ggplot2 for data
visualisation, check Chapter 3 of Wickham and Grolemund (2020), the
RStudio cheatsheets (RStudio 2012), and Chang (2017).

\hypertarget{references}{%
\section{References}\label{references}}

\footnotesize

\hypertarget{refs}{}
\begin{CSLReferences}{1}{0}
\leavevmode\vadjust pre{\hypertarget{ref-gapminder}{}}%
Bryan, Jennifer. 2017. \emph{Gapminder: Data from Gapminder}.
\url{https://CRAN.R-project.org/package=gapminder}.

\leavevmode\vadjust pre{\hypertarget{ref-changgraphics}{}}%
Chang, Winston. 2017. \emph{R Graphics Cookbook: Practical Recipes for
Visualizing Data}. 2nd ed. O'Reilly Media.
\url{http://www.cookbook-r.com/Graphs/}.

\leavevmode\vadjust pre{\hypertarget{ref-Friendly:05:gfkl}{}}%
Friendly, M. 2005. {``Milestones in the History of Data Visualization: A
Case Study in Statistical Historiography.''} In \emph{Classification:
The Ubiquitous Challenge}, edited by C. Weihs and W. Gaul, 34--52. New
York: Springer. \url{http://www.math.yorku.ca/SCS/Papers/gfkl.pdf}.

\leavevmode\vadjust pre{\hypertarget{ref-Friendly:06:hbook}{}}%
---------. 2006. {``A Brief History of Data Visualization.''} In
\emph{Handbook of Computational Statistics: Data Visualization}, edited
by C. Chen, W. Härdle, and A Unwin. Vol. III. Heidelberg:
Springer-Verlag. \url{http://www.datavis.ca/papers/hbook.pdf}.

\leavevmode\vadjust pre{\hypertarget{ref-pantoliano}{}}%
Pantoliano, Mike. 2012. {``Data Visualization Principles: Lessons from
Tufte.''} 2012.
\url{https://moz.com/blog/data-visualization-principles-lessons-from-tufte}.

\leavevmode\vadjust pre{\hypertarget{ref-cheatsheets}{}}%
RStudio. 2012. {``RStudio Cheat Sheets.''} 2012.
\url{https://www.rstudio.com/resources/cheatsheets/}.

\leavevmode\vadjust pre{\hypertarget{ref-tufte}{}}%
Tufte, Edward R. 1983. \emph{The Visual Display of Quantitative
Information}. Graphics Press.

\leavevmode\vadjust pre{\hypertarget{ref-visstatthink1997}{}}%
---------. 1997. {``Visual and Statistical Thinking.''} In \emph{Visual
Explanations: Images and Quantities, Evidence and Narrative}. Graphics
Press. \url{https://www.edwardtufte.com/tufte/books_textb}.

\leavevmode\vadjust pre{\hypertarget{ref-r4ds}{}}%
Wickham, Hadley, and Garrett Grolemund. 2020. \emph{R for Data Science:
Import, Tidy, Transform, Visualize, and Model Data}.
\url{http://r4ds.had.co.nz}.

\end{CSLReferences}

\end{document}
